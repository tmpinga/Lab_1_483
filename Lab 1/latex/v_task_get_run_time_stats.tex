task. h 
\begin{DoxyPre}void vTaskGetRunTimeStats( char *pcWriteBuffer );\end{DoxyPre}


config\-G\-E\-N\-E\-R\-A\-T\-E\-\_\-\-R\-U\-N\-\_\-\-T\-I\-M\-E\-\_\-\-S\-T\-A\-T\-S must be defined as 1 for this function to be available. The application must also then provide definitions for \hyperlink{_free_r_t_o_s_8h_a727939bcdb98501e0eba0ec8a1841e1b}{port\-C\-O\-N\-F\-I\-G\-U\-R\-E\-\_\-\-T\-I\-M\-E\-R\-\_\-\-F\-O\-R\-\_\-\-R\-U\-N\-\_\-\-T\-I\-M\-E\-\_\-\-S\-T\-A\-T\-S()} and port\-G\-E\-T\-\_\-\-R\-U\-N\-\_\-\-T\-I\-M\-E\-\_\-\-C\-O\-U\-N\-T\-E\-R\-\_\-\-V\-A\-L\-U\-E to configure a peripheral timer/counter and return the timers current count value respectively. The counter should be at least 10 times the frequency of the tick count.

N\-O\-T\-E\-: This function will disable interrupts for its duration. It is not intended for normal application runtime use but as a debug aid.

Setting config\-G\-E\-N\-E\-R\-A\-T\-E\-\_\-\-R\-U\-N\-\_\-\-T\-I\-M\-E\-\_\-\-S\-T\-A\-T\-S to 1 will result in a total accumulated execution time being stored for each task. The resolution of the accumulated time value depends on the frequency of the timer configured by the \hyperlink{_free_r_t_o_s_8h_a727939bcdb98501e0eba0ec8a1841e1b}{port\-C\-O\-N\-F\-I\-G\-U\-R\-E\-\_\-\-T\-I\-M\-E\-R\-\_\-\-F\-O\-R\-\_\-\-R\-U\-N\-\_\-\-T\-I\-M\-E\-\_\-\-S\-T\-A\-T\-S()} macro. Calling \hyperlink{task_8h_ac34910d5eac69f0538ee218e527663a7}{v\-Task\-Get\-Run\-Time\-Stats()} writes the total execution time of each task into a buffer, both as an absolute count value and as a percentage of the total system execution time.


\begin{DoxyParams}{Parameters}
{\em pc\-Write\-Buffer} & A buffer into which the execution times will be written, in ascii form. This buffer is assumed to be large enough to contain the generated report. Approximately 40 bytes per task should be sufficient. \\
\hline
\end{DoxyParams}
